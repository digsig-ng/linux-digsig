\documentclass{article}
\usepackage{graphics,epsfig,url,times}
\begin{document}

%don't want date printed
%\date{}

\pagestyle{empty}


\title{\Large \bf The DigSig project}

\author{
A. Apvrille - {\normalsize Axelle.Apvrille@trusted-logic.fr}\\
D. Gordon - {\normalsize David.Gordon@ericsson.ca}\\
S. Hallyn - {\normalsize serue@us.ibm.com}\\
M. Pourzandi - {\normalsize Makan.Pourzandi@ericsson.ca}\\
V. Roy - {\normalsize Gaspoucho@yahoo.com}\\
} % end author

\maketitle

%\thispagestyle{empty}

This working documentation presents the DigSig project, 
a Linux kernel module capable of verifying
digital signatures of ELF binaries before running them. This kernel
module is available under the GPL license at 
http://sourceforge.net/projects/disec/, and has been successfully
tested for kernel 2.5.66 and above.

\section{Why Check the Signature of Your Binaries Before Running Them ?}

The problem with blindly running executables is that you are never
sure they actually do what you think they are supposed to do (and
nothing more...): if viruses spread so much on Microsoft Windows
systems, it is mainly because users are frantic to execute whatever
they receive, especially if the title is appealing... The LoveLetter
virus, with over 2.5 million machines infected, is a famous
illustration of this.  Yet, Linux is unfortunately not immune to
malicious code either \cite{linux-world}. By executing unknown and untrusted code,
users are exposed to a wide range of Unix worms, viruses, trojans,
backdoors etc.  To prevent this, a possible solution is to digitally
sign binaries you trust, and have the system check their digital
signature before running them: if the signature cannot be verified, the
binary is declared corrupt and operating system will not let it run.

\section{Related Work} 

There has already been several initiatives in this domain,
such as Tripwire, BSign, Cryptomark, and IBM's Signed Executables \cite{Tripwire, Bsign, Cryptomark, IBM-sign}
but we believe the DigSig project is the first to be both 
easily accessible to all (available on Sourceforge, 
under the GPL license) and to operate at kernel level (see Tab. 1).


\begin{table}[ht]
\begin{center}
\begin{tabular}{|p{2cm}|l|l|l|l|}
\hline
 & Real-time signature verification & File type & Level & Availability \\ 
\hline
Tripwire    & No           & All       & User      & Commercial  \& GPL \\ 
\hline
Cryptomark  & Yes          & Binaries  & Kernel    & CryptoMark's web: "Not  \\& & & &  ready for public release" \\
\hline
Signed  Executables    & Yes          & Binaries \& scripts & Kernel    & Not GPL   \\     
\hline
DigSig      & Yes          & Binaries  & Kernel    & GPL \\ 
\hline
\end{tabular}
\caption{Comparison between file signing tools.}
\label{tab:userspace}
\end{center}
\end{table}

%% \begin{verbatim} 
%% +-------------+--------------+-----------+-----------+--------------+
%% +             + Real-time    + File type + Level     + Availability +
%% +             + signature    +           +           +              +
%% +             + verification +           +           +              +
%% +-------------+--------------+-----------+-----------+--------------+
%% + Tripwire    + No           + All       + User      + Commercial & +
%% +             +              +           +           + GPL          +
%% +-------------+--------------+-----------+-----------+--------------+
%% + BSign       + No           + Binaries  + User      + GPL          +
%% +-------------+--------------+-----------+-----------+--------------+
%% + Cryptomark  + Yes          + Binaries  + Kernel    + From Crypto  +
%% +             +              +           +           + Mark's web:  +
%% +             +              +           +           + "Not ready   +
%% +             +              +           +           + for public   +
%% +             +              +           +           + release"     +
%% +-------------+--------------+-----------+-----------+--------------+
%% + Signed      + Yes          + Binaries  + Kernel    + Not GPL      + 
%% + Executables +              + + scripts +           +              +
%% +-------------+--------------+-----------+-----------+--------------+
%% + DigSig      + Yes          + Binaries  + Kernel    + GPL          +
%% +-------------+--------------+-----------+-----------+--------------+
%% Tab 1. Comparison between file signing tools

%% \end{verbatim} 

\section{The DigSig Solution}

In order to avoid re-inventing the wheel, we based our solution on the
existing open source project BSign: a Debian userspace binary
signing package. BSign signs the binaries and embeds the signature in
the binary itself. Then, at kernel level, DigSig verifies these 
signatures at execution time and denies execution if signature is invalid. 

Typically, in our approach, binaries are not signed by vendors, but we
rather hand over control of the system to the local administrator.
He/she is responsible to sign all binaries he/she trusts with his/her
private key. Then, those binaries are verified with the corresponding
public key. This means you can still use your favorite (signed)
binaries: no change in habits. Basically, DigSig only guarantees two
things: (1) if you signed a binary, nobody else than you can modify
that binary without being detected, and (2) nobody can run a binary
which is not signed or badly signed.  Of course, you should be careful
not to sign untrusted code: if malicious code is signed, all security
benefits are lost.

\section{How do I use DigSig ?}
DigSig is fairly simple to use. First, you need to sign all binaries
you trust with BSign (version 0.4.5 or more). Then you need to load
DigSig with the public key which corresponds to the private key
used to sign the binaries.

In the following we show step by step how to sign the executable "ps":

\begin{verbatim} 
$ cp `which ps` ps-test
$ bsign -s ps-test // Sign the binary
$ bsign -V ps-test // Verify the validity of the signature 
\end{verbatim} 

Then, you need to install the DigSig kernel module. To do so, a recent
kernel version is required (25.66 or more), compiled with security
options enabled (CONFIG\_SECURITY$=$y).  To compile DigSig, assuming your
kernel source directory is /usr/src/linux-2.5.66, you do:

\begin{verbatim} 
$ cd digsig
$ make -C /usr/src/linux-2.5.66 SUBDIRS=$PWD modules
$ cd digsig/tools && make
\end{verbatim} 

This builds the DigSig kernel module (digsig\_verifko), and you are
probably already half-way through the command to load it, but wait !
If you are not cautious about the following point, you might secure
your machine so hard you'll basically freeze it. As a matter of fact,
once DigSig is loaded, verification of binary signatures is activated.
At that time, binaries will be able to run only if their signature is
successfully verified. In all other cases (invalid signature,
corrupted file, no signature...), execution of the binary will be
denied. Consequently, if you forget to sign an essential binary such
as /sbin/reboot, or /sbin/rmmod, you'll be most embarrassed to reboot
the system if you have to...  Therefore, for testing purposes, we
recommend you initially run DigSig in debug mode. To do this, compile
DigSig with the DSI\_DIGSIG\_DEBUG and DSI\_DEBUG flags set in the
Makefile:

\begin{verbatim} 
EXTRA_CFLAGS += -DDSI_DEBUG -DDSI_DIGSIG_DEBUG -I $(obj)
\end{verbatim} 

In debug mode, DigSig lets unsigned binaries run. This state is ideal
to test DigSig, and also list the binaries you need to sign to get a
fully operational system.

Once this precaution has been taken, it is now time to load the DigSig
module, with your public key as argument. BSign uses GnuPG keys to
sign binaries, so retrieve your public key as follows:

\begin{verbatim} 
$ gpg --export >> my_public_key.pub 
\end{verbatim} 

Then, log as root, and use the digsig.init script to load the module.

\begin{verbatim} 
# ./digsig.init start my_public_key.pub
Testing if sysfs is mounted in /sys.
sysfs found
Loading Digsig module.
Loading public key.
Done.
\end{verbatim} 

This is it: signature verification are activated. You can check the 
signed ps executable (ps-test) works:

\begin{verbatim} 
$./ps-test
$ su
Password:
# tail -f /var/log/messages
colby kernel: DSI-LSM MODULE - binary is ./ps-test
colby kernel: DSI-LSM MODULE - dsi_bprm_compute_creds: Found signature 
                                                                 section
colby kernel: DSI-LSM MODULE - dsi_bprm_compute_creds: Signature verification 
                                                                 successful
\end{verbatim} 

But, corrupted executables won't run:
\begin{verbatim} 
$ ./ps-corrupt
bash: ./ps-corrupt: Operation not permitted 
colby kernel: DSI-LSM MODULE - binary is ./ps-corrupt
colby kernel: DSI-LSM MODULE Error - dsi_bprm_compute_creds: Signatures 
                                         do not match for ./ps-corrupt
\end{verbatim} 

If the permissive debug mode is set, signature verification is skipped
for unsigned binaries.  Otherwise, the control is strictly enforced in
the normal behavior:

\begin{verbatim} 
$ ./ps
bash: ./ps: cannot execute binary file 
# su 
Password: 
# tail -f /var/log/messages
colby kernel: DSI-LSM MODULE - binary is ./ps
colby kernel: DSI-LSM MODULE - dsi_bprm_compute_creds: Signatures 
                                                        do not match
\end{verbatim} 

\section{DigSig, behind the scene}

The core of DigSig lies in the LSM hooks placed in the kernel's
routines for executing a binary. The starting point of any binary
execution is a system call to sys\_exec() which triggers
do\_execve(). This is the transition between user space and kernel
space.

The first LSM hook to be called is bprm\_alloc\_security,
where a security structure is optionally attached to the
linux\_bprm structure which represents the task. DigSig
does not use this hook as it doesn't need any specific
security structure.
Then, the kernel tries to find a binary handler 
(search\_binary\_handler) to load the file. This is
when the LSM hook bprm\_check\_security is called, and precisely
when DigSig performs signature verification of the binary
If successful, load\_elf\_binary() gets called,
which eventually calls do\_mmap(), then the LSM hook
file\_mmap(), and finally bprm\_free\_security().

%<< include System-call.jpg >>
%Fig.1: Control flow in binary execution

\begin{figure}[h]
\begin{center}
\epsfxsize=14cm 
\epsfbox{System-call}
\caption{{\it Control flow in binary execution. }}
\label{fig1}
\end{center}
\end{figure}

So, this is how DigSig enforces binary signature verification at
kernel level. Now, let's shortly explain the signing mechanism of
DigSig's userland counterpart: BSign. When signing an ELF binary,
BSign stores the signature in a new section in the binary.  To do so,
it modifies the ELF's section header table to account for this new
section, with the name 'signature' and a user defined type 0x80736967
(which comes from the ASCII characters 's', 'i' and 'g'). You can
check your binary's section header table with the command readelf -S
binary.  Then, it performs a SHA1 hash on the entire file, after
having zeroed the additional signature section.  Next, it prefixes
this hash with "\#1; bsign v\%s" where \%s is the version number of
BSign, and stores the result at the begining of the binary's signature
section. Finally, BSign calls GnuPG to sign the signature section
(containing the hash), and stores the signature at the current position
of the signature section. A short compatibility note: GnuPG adds a
32-byte timestamp and a signature class identifier in the
buffer it signs.

\begin{figure}[h]
\begin{center}
\epsfxsize=10cm 
\epsfbox{ELF-Format}
\caption{{\it BSign's signature section as added in an ELF binary. }}
\label{fig2}
\end{center}
\end{figure}

%%<< include ELF-format.jpg >> 
%%Fig.2: BSign's signature section as added in an ELF binary 

On a cryptographic point of view, DigSig needs to verify 
BSign's signatures, i.e RSA signatures. More precisely, this 
consists in, on one side, hashing the binary
with a one-way function (SHA-1) and padding the result (EMSA PKCS1 v1.5), and,
on the other side, "decrypting" the signature with the public key
and verifying this corresponds to the padded text.

PKCS\#1 padding is pretty simple to implement, so we had no problems
coding it. Concerning SHA-1 hashing, we used Linux's kernel CryptoAPI:

\begin{itemize} 
\item we allocate a crypto\_tfm structure (crypto\_alloc\_tfm), and use
  it to initialize the hashing process (crypto\_digest\_init)
\item then, we read the binary block by block, and feed it to the hashing
  routine (crypto\_digest\_update)
\item finally, we retrieve the hash (crypto\_digest\_final).
\end{itemize} 
The trickiest part is most certainly the RSA verification because
the CryptoAPI does not support asymetric algorithms (such as RSA) yet,
so we had to implement it... The theory behind RSA is relatively
simple: it consists in a modular exponentation ($m^e mod n$) using
very large primes, however, in practice, everybody will agree that 
implementing an efficient big number library is tough work. 
So, instead of writing ours, we decided it would be safer to use
an existing one and adapt it to kernel restrictions.
We decided to port GnuPG's math library (which is
actually derived from GMP, GNU's math library) \cite{GnuPG}:
\begin{itemize} 
\item only the RSA signature verification routines have been kept. 
  For instance, functions to generate large primes have been erased.
\item allocations on the stack have been limited to the strict minimum.
\end{itemize} 

\section{How much does it slow the system down ?} 

We have performed two different kinds of benchmarks for DigSig: 
a benchmark of the real impact of DigSig for users (how much they
feel the system is slowed down), and a more precise benchmark
evaluating the exact overhead induced by our kernel module.

The first set of benchmarks have been performed by comparing
how long it takes to run an executable with or without DigSig.
To do so, we used the command 'time' over fast to longer executions. 
The following benchmark has been run 20 times:

\begin{verbatim} 
% time /bin/ls -Al		  # times /bin/ls
% time ./digsig.init compile      # times compilation with gcc
% time tar jxvfp linux-2.6.0-test8.tar.bz2 # times tar
\end{verbatim} 

On a Pentium 4, 2.2 Ghz, with 512 MB of RAM, with DigSig using GnuPG's
math library, we have obtained results displayed at Table 2. They
clearly show that the impact of DigSig is quite important for short
executions (such as ls) but soon becomes completely negligeable for
longer executions such as compiling a project with gcc, or untarring
sources with tar.


\begin{table}[ht]
\begin{center}
\begin{tabular}{|l|l|l|l|}
\hline
 & ls & gcc   & tar    \\ 
\hline
Without DigSig & 0.006 & 15.727 & 36.7345 \\  
\hline
With DigSig    & 0.008 & 16.550 & 36.7333 \\
\hline
\end{tabular}
\caption{Average execution time in seconds with or without DigSig.}
\label{tab:userspace}
\end{center}
\end{table}


%% \begin{verbatim} 
%% +----------------+-------+--------+--------+
%% |                | ls    | gcc    | tar    |
%% +----------------+-------+--------+--------+
%% | Without DigSig | 0.006 | 15.727 | 36.7345| 
%% | With DigSig    | 0.008 | 16.550 | 36.7333|
%% +----------------+-------+--------+--------+
%% Tab 2. Average execution time in seconds with or without DigSig.
%% \end{verbatim} 

Second, we have measured the exact overhead introduced by our kernel
module. To do so, we have basically compared jiffies at the beginning
and at the end of bprm\_check\_security. In brief, jiffies represent
the number of clock ticks since the system has booted, so they
are a precise way to measure time in the Linux kernel. In our case,
jiffies are in milliseconds. We have run each binary 30 times
(see Tab 3) for DigSig compiled with GnuPG.

\begin{table}[ht]
\begin{center}
\begin{tabular}{|l|l|l|}
\hline
Binaries  & Size  &  GnuPG   \\
\hline 
ls &   68230 &  1.61765 \\ 
\hline 
ps &   70337 &  1.67742 \\
\hline 
busybox & 248801 &  4.66666 \\
\hline 
cvs &  672532 & 11.53571 \\
\hline 
vim & 1894305 & 31.25000 \\
\hline 
emacs & 4093614 & 67.41176 \\
\hline 
\end{tabular}
\caption{Average DigSig overhead, in milliseconds, for various 
binaries.}
\label{tab:userspace}
\end{center}
\end{table}


%% \begin{verbatim} 
%% +-----------+---------+----------+
%% | Binaries  |   Size  |  GnuPG   |
%% +-----------+---------+----------+
%% |        ls |   68230 |  1.61765 |
%% |        ps |   70337 |  1.67742 |
%% |   busybox |  248801 |  4.66666 |
%% |       cvs |  672532 | 11.53571 |
%% |       vim | 1894305 | 31.25000 |
%% |     emacs | 4093614 | 67.41176 |
%% +-----------+---------+----------+
%% Tab 3. Average DigSig overhead, in milliseconds, for various 
%% binaries.
%% \end{verbatim} 

The results show that, naturally, the digital signature verification
overhead increases with executable's size (which is not a surprise
because it takes longer to hash all data).

Finally, to assist us in optimizing our code, we have run Oprofile
\cite{OProfile}, a system profiler for Linux, over DigSig (see Figure 3). Results
clearly indicate that the modular exponentiation routines are the most
expensive, so this is where we should concentrate our optimization
efforts for future releases. More particularly, we plan to port ASM
code of math libraries to the kernel, instead of using pure C code.

\begin{verbatim} 
CPU: CPU with timer interrupt, speed 2398.91 MHz (estimated)
Profiling through timer interrupt
vma      samples  %           image name       app name     symbol name
00001150 401      42.5239     digsig_verifko  digsig_verif mpihelp_submul_1
00001090 198      20.9968     digsig_verif.ko  digsig_verif mpihelp_addmul_1
00000e90 109      11.5589     digsig_verif.ko  digsig_verif dsi_sha1_update
00002fa0 77        8.1654     digsig_verif.ko  digsig_verif mpihelp_divrem
00000fd0 32        3.3934     digsig_verif.ko  digsig_verif mpihelp_mul_1
00001300 27        2.8632     digsig_verif.ko  digsig_verif mpihelp_add_n
00001290 15        1.5907     digsig_verif.ko  digsig_verif mpihelp_sub_n

                    Fig 3. OProfile report for DigSig. 
\end{verbatim} 

\section{Caching}

Digsig impacts performance only at the beginning of file execution.  For
long-lived applications which are executed once, such as mozilla, the
amortized cost is likely acceptable.  However, the cost of repeatedly
checking signatures on the same executables (such as {\tt ls}) and
libraries (such as {\tt libc}) can become significant depending upon
the workload.

To combat this, digsig keeps a cache of validated signature checks
When a file's signature has been validated, its inode is added into
a hash table.  The next time the file is loaded, its presence in
this hash table will serve as signature validation without requiring
recomputation of the signature.

Caching signature validations can be risky.  We must ensure that an
attacker cannot use this feature to cause an altered version of a file
to be loaded without the (now invalid) signature being checked.  In
the simplest, case, a new file is copied in place of the validated file.
Since Digsig caches decisions based on the inode, and the new file will
have a different
inode than the old file, the signature will be computed and checked for the
new file.  If, instead, a process attempts to write to an existing file
whose signature validation has been cached, then the signature
validation will be cleared.  The next time a process executes this file,
the signature will be recomputed.  Finally, if a process is still
executing a file while another process attempts to write to it, the
Linux kernel will deny the request for write access.

There is still a risk, however, of the file being overwritten at a lower
layer than the VFS.  In particular, this could happen with files mounted
over NFS.  In this case, we would depend on the NFS server to deny write
requests from untrustworthy clients.

Caching will not affect the speed of computation for a particular
signature.  Its effect will be most dramatic while doing many quick
repeated executions.  An example of such a workload is compilation of
large packages, which repeat the same sequence of actions on many different
files.  To measure a best case performance improvement of
caching, we timed compilation of Digsig itself in three ways.  Using a kernel
without the Digsig module loaded, with digsig loaded but caching
disabled, and with digsig using caching.

Figures~\ref{fig:tartimings}, \ref{fig:lstimings}, and \ref{fig:comptimings} show
the amount of time
each of these three systems required to untar the kernel source, perform a
directory listing on the top level of the kernel source, and to compile the
actual kernel.  Care was taken to compile all kernels with the same configuration.
Each test was performed three times.

\begin{figure}
\begin{tabular}{|l|r|r|r|}
\hline
	Kernel without Digsig & & & \\
	real  &  0m59.937s &0m59.175s &0m58.493s \\
	user  &  0m42.058s &0m42154s &0m42.225s \\
	sys   &  0m4.005s  &0m3.939s  &0m3.895s \\
\hline
	Digsig without caching & & & \\
	real  &  1m0.405s  & 0m59.361s & 0m59.329s \\
	user  &  0m42.269s & 0m42.226s & 0m42.190s \\
	sys   &  0m3.981s  & 0m3.927s  & 0m4.005s \\
\hline
	Digsig with caching & & & \\
	real  &  0m59.660s & 0m59.827s & 0m59.724s \\
	user  &  0m42.178s & 0m42195s & 0m42.120s \\
	sys   &  0m4.008s  & 0m3.921s  & 0m3.940s \\
\hline
\end{tabular}
\caption{Time required for ``tar jxvfp linux-2.6.0-test8.tar.bz2''}\label{fig:tartimings}
\end{figure}

\begin{figure}
\begin{tabular}{|l|r|r|r|}
\hline
	Kernel without Digsig & & & \\
	real  &  0m0.065s &0m0.007s &0m0.006s \\
	user  &  0m0.001s &0m0.002s &0m0.002s \\
	sys   &  0m0.005s &0m0.003s &0m0.003s \\
\hline
	Digsig without caching & & & \\
	real  &  0m0.049s &0m0.053s &0m0.048s \\
	user  &  0m0.003s &0m0.001s &0m0.003s \\
	sys   &  0m0.044s &0m0.042s &0m0.043s \\
\hline
	Digsig with caching & & & \\
	real  &  0m0.025s& 0m0.006s &0m0.006s \\
	user  &  0m0.001s& 0m0.000s &0m0.003s \\
	sys   &  0m0.005s& 0m0.003s &0m0.004s \\
\hline
\end{tabular}
\caption{Time required for ``/bin/ls -Al''}\label{fig:lstimings}
\end{figure}

\begin{figure}
\begin{tabular}{|l|r|r|r|}
\hline
	Kernel without Digsig & & & \\
	real   & 0m22.836s& 0m15.716s &0m15.700s \\
	user   & 0m14.291s& 0m14.207s &0m14.242s \\
	sys    & 0m1.449s &0m1.461s &0m1.427s \\
\hline
	Digsig without caching & & & \\
	real   & 0m42.597s& 0m32.629s &0m32.412s \\
	user   & 0m14.577s& 0m14.513s &0m14.501s \\
	sys    & 0m16.073s& 0m16.112s &0m16.158s \\
\hline
	Digsig with caching & & & \\
	real   & 0m22.996s& 0m15.636s &0m15.612s \\
	user   & 0m14.167s& 0m14.179s &0m14.108s \\
	sys    & 0m1.543s &0m1.408s   & 0m1.477s \\
\hline
\end{tabular}
\caption{Time required for ``/bin/ls -Al''}\label{fig:comptimings}
\end{figure}

Since the signature validation occurs during {\tt execve}, we expect its
time to show up in system, or {\tt sys} time.
The least impact was seen in the {\tt tar} operation.  This is because
we performed many file creations, which also appear under system time.
In contrast, {\tt tar} was a single execution, requiring only one signature
validation.  Therefore the file operations effectively masked the signature
validation check.  The
impact of the signature check is more dramatic in the other two tests,
where Digsig without caching is eight to fourteen times slower than
Digsig with caching, or a kernel without Digsig.  The latter two
performed effectively the same, with Digsig with caching sometimes
outperforming a Digsig-free kernel.

Finally, compilation of a full kernel required 592 seconds without
Digsig, 588 seconds with caching, and 1029 with digsig but without caching.
Caching of signature validations manages very effectively
eliminate the performance impact of Digsig under what would ordinarily
be its worst workloads.

\section{Digital Signature of Shared Libraries}

TO WRITE.

\subsubsection*{End remark}

{\small This document has been {\it partly} published in Linux World, vol. 2., no.1, January 2004.}

\begin{thebibliography}{99}

\bibitem{linux-world} 
  {\em Wraight C., Securing your linux environment, Linux World, Vol 1, 
    Issue 2, pages 48-51, November/December 2003.}.

\bibitem{Tripwire} 
  {\em Tripwire, http://www.tripwire.com}.

\bibitem{Bsign} 
  {\em Bsign, http://packages.qa.debian.org/b/bsign.html}. 

\bibitem{Cryptomark} 
  {\em Cryptomark, http://www.immunix.org/cryptomark.html}.

\bibitem{IBM-sign}
  {\em Van Doorn, L., Ballintijn, G., Arbaugh, W.A., Signed Executables for Linux, January 2003.}.

\bibitem{GnuPG}
  {\em GnuPG, http://www.gnupg.org}. 

\bibitem{OProfile}
  {\em OProfile, http://oprofile.sourceforge.net}. 


\end{thebibliography}

\end{document}
